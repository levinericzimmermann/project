\documentclass[12pt,a4paper,british,landscape]{article}

\usepackage[fontsize=22.01pt]{fontsize}
\usepackage{enumerate}


\renewcommand*\ttdefault{cmvtt}
\renewcommand{\familydefault}{\ttdefault}

\usepackage[OT2,T1]{fontenc}

\renewcommand\labelitemi{---}


\usepackage{graphicx}
\usepackage[
    a4paper,
    bindingoffset=0cm,
    left=1cm,
    right=1cm,
    top=0.5cm,
    bottom=0.2cm,
    footskip=0cm
]{geometry}

\frenchspacing              % Better looking spacings after periods
\pagestyle{empty}           % No pagenumbers/headers/footers



\begin{document}





the succession of apparent alterations in the form of the moon presents a phenomenon so remarkable as necessarily to have attracted the attention and careful observation of man from the earliest period. with the greeks the phases were named;—

\begin{enumerate}[I]
\setlength\itemsep{0.02cm}

    

\item \small the new moon. noumênia, which because in the same line or path with the sun, is called synodos.

\item the young moon. nea selênê. time in the month, protê phasis, 'the first appearance;' a slender crescent seen a short time after sunset.

\item the increasing crescent. hexagônos, 'six-angled,' as having run 1/6th of its course.

\item the half moon. hemitomos, 'cut-in-twain.' 1 also called tetragônos, as having four equal angles in its circuit, ¼th of which it has now passed.

\item the increasing moon. amphikurtos, 'curved- on-each-side.' also called trigônos, 'triangular,' for were an equilateral triangle drawn from its starting-point, the present position would be the apex, ⅓rd of its course being now passed.

\item the full moon. panselênos. also called dichomênia, the 'month-divider.'

\item  the decreasing moon. amphikurtos, trigonos.

\item the second half-moon. hemitomos, etc.

\item the decreasing crescent. menoeidês, 'crescent-shaped,' lat. lunatus.

\item  the old moon. enê selênê. time in the month,—eschatê phasis, 'the last appearance.' a slender crescent.

\end{enumerate}

\vspace{-0.6cm}

\begin{center}
(robert brown. the unicorn, a mythological investigation. 1881)
\end{center}


\end{document}


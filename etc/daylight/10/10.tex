\documentclass[12pt,a4paper,british,landscape]{article}

\usepackage[fontsize=27.31pt]{fontsize}

\renewcommand*\ttdefault{cmvtt}
\renewcommand{\familydefault}{\ttdefault}

\usepackage[OT2,T1]{fontenc}

\renewcommand\labelitemi{---}


\usepackage{graphicx}
\usepackage[
    a4paper,
    bindingoffset=0cm,
    left=1cm,
    right=1cm,
    top=1.2cm,
    bottom=0cm,
    footskip=0cm
]{geometry}

\frenchspacing              % Better looking spacings after periods
\pagestyle{empty}           % No pagenumbers/headers/footers



\begin{document}




\begin{center}


\vspace{0.2cm}

``the act of making music, clothes, art, or even food has a very different, and possibly more beneficial effect on us than simply consuming those things.
and yet for a very long time, the attitude of the state toward teaching and funding the arts has been in direct opposition to fostering creativity among the general population.
it can often seem that those in power don't want us to enjoy making things for ourselves -- they'd prefer to establish a cultural hierarchy that devalues our amateur efforts and encourages consumption rather than creation.
this might sound like i believe there is some vast conspiracy at work, which i don't, but the situation we find ourselves in is effectively the same as if there were one.
the way we are taught about music, and the way it’s socially and economically positioned, affect whether it's integrated (or not) into our lives, and even what kind of music might come into existence in the future.
capitalism tends toward the creation of passive consumers, and in many ways this tendency is counterproductive.''

\vspace{0.5cm}

(david byrne. how music works. canongate. 2012)


\end{center}

\end{document}


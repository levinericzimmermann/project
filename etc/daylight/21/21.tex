\documentclass[12pt,a4paper,british,landscape]{article}

\usepackage{anyfontsize}


\renewcommand*\ttdefault{cmvtt}
\renewcommand{\familydefault}{\ttdefault}

\usepackage[OT2,T1]{fontenc}

\renewcommand\labelitemi{---}


\usepackage{graphicx}
\usepackage[
    a4paper,
    bindingoffset=0cm,
    left=1cm,
    right=1cm,
    top=1.5cm,
    bottom=0.2cm,
    footskip=0cm
]{geometry}

\frenchspacing              % Better looking spacings after periods
\pagestyle{empty}           % No pagenumbers/headers/footers



\begin{document}

{\fontsize{1.115cm}{1.2cm}\selectfont

``society runs the risk of moving at two speeds [\dots].
the slow group are those who have been socially left behind.
in an interview with an unemployed youth, which was recorded in february 1984 in bremen [\dots] the ``standstill'' of time in the social situation of being ``unemployed'' is made plain.
``tiger'', as the youth calls himself, wants to make time stand still because ``there must not be no such a bad time in his life''.
his strategies are directed towards making this time irrelevant [\dots].
in contrast to the bold promise of one commentator at the turn of the century, space and time have not been annihilated [\dots].
the progress of humanity conceived in evolutionary terms still compares itself with the arrow of time which points irreversibly forwards.''
% but far more than this become apparent.
% a standstill means a relapse in time.
% tiger and his family -- his father is a dockworker -- live in a temporally different [\dots] world.''

\vspace{0.25cm}

\begin{center}
(helga nowotny)
\end{center}

}


\end{document}


\documentclass[12pt,a4paper,british,landscape]{article}

\usepackage{anyfontsize}


\renewcommand*\ttdefault{cmvtt}
\renewcommand{\familydefault}{\ttdefault}

\usepackage[OT2,T1]{fontenc}

\renewcommand\labelitemi{---}


\usepackage{graphicx}
\usepackage[
    a4paper,
    bindingoffset=0cm,
    left=1cm,
    right=1cm,
    top=0.8cm,
    bottom=0.2cm,
    footskip=0cm
]{geometry}

\frenchspacing              % Better looking spacings after periods
\pagestyle{empty}           % No pagenumbers/headers/footers



\begin{document}

{\fontsize{0.915cm}{1cm}\selectfont

``it is important to read time as a situated construction, not a natural, neutral, or universal medium through which all people move in the same way, because colonialism works by normalizing its practices as universal and innate [\dots].
modern temporality is like a grammar of time, an underlying structure that links and organizes multiple iterations of colonial time.
site-specific, socially situated iterations of time, in contrast, are like discourses, and i refer to them as “forms of time.”
discursive forms of time are inflected by the specific places, people, and histories involved in their construction.
colonial forms of time may contradict one another, but they all maintain their ideological power through recourse to modernity’s supposedly universal temporality of homogenous, empty, linear time.
for example, u.s. railroad companies’ creation of time zones depended on modernity’s empty, homogenous time to arbitrarily map time onto space.
likewise, specific narratives of development—psychological, sociological, and biological—depend on modern time’s linearity to prescribe a normative path of progression that authorizes certain peoples, individuals, and places as more (or less) advanced than others.''

\vspace{0.25cm}

\begin{center}
(erin murrah--mandril. in the mean time. 2020)
\end{center}

}


\end{document}


\documentclass[12pt,a4paper,british,landscape]{article}

\usepackage[fontsize=24.31pt]{fontsize}

\renewcommand*\ttdefault{cmvtt}
\renewcommand{\familydefault}{\ttdefault}

\usepackage[OT2,T1]{fontenc}

\renewcommand\labelitemi{---}


\usepackage{graphicx}
\usepackage[
    a4paper,
    bindingoffset=0cm,
    left=1cm,
    right=1cm,
    top=1.2cm,
    bottom=0cm,
    footskip=0cm
]{geometry}

\frenchspacing              % Better looking spacings after periods
\pagestyle{empty}           % No pagenumbers/headers/footers



\begin{document}




\begin{center}


\vspace{0.2cm}

``[\dots] william grossin notes, `there is a correspondence, a correlation between the economy of a society, the way on which labour is organized, the means used for the production of its goods and service and the representation of time in the collective consciousness'.
in societies that did not measure and ‘keep’ time, however, europeans perceived a kind of status naturae as prevailing, since the degree of separation of human-time from nature’s time operated as a measure of a society’s progress in the quest to transcend natural limitations.
    hunter-gatherer rituals and nomadic lifestyles were labelled as primitive and savage partly because they were seen as being guided not by a rational, linear and man-made calendar and clock but by ‘unpredictable’ and ‘irregular’ cues dictated by the natural environment: the rising of a specific star, the phases of the moon or the seasonal appearance of flora and fauna [\dots].
the western humanist conception of time is aptly conveyed in the workings of the clock and the seven-day week -- two quintessentially man-made inventions whose claim to superiority in their colonial observers' eyes, was their apparent abstraction and separation from the rhythms of nature.''

\vspace{0.5cm}

(giordano nanni. the colonisation of time. manchester. 2012)


\end{center}

\end{document}


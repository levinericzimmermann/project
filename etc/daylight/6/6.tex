\documentclass[12pt,a4paper,british,landscape]{article}

\usepackage[fontsize=22.01pt]{fontsize}

\renewcommand*\ttdefault{cmvtt}
\renewcommand{\familydefault}{\ttdefault}

\usepackage[OT2,T1]{fontenc}

\renewcommand\labelitemi{---}


\usepackage{graphicx}
\usepackage[
    a4paper,
    bindingoffset=0cm,
    left=1cm,
    right=1cm,
    top=0.5cm,
    bottom=0cm,
    footskip=0cm
]{geometry}

\frenchspacing              % Better looking spacings after periods
\pagestyle{empty}           % No pagenumbers/headers/footers



\begin{document}




\begin{center}


[\dots] in other words people, manufactured objects and things such as trees are not distinct categories based on biology or the possession of life.
rocks, trees and animals are all examples of material culture and as such can be part of relational networks, as well as relating to each other independently of people.
for example, a hen-house is built by people.
but the hens that live in it have a relation to those surroundings which conditions their actions when the chicken farmer is far away [\dots]
gell provided an answer for one category of material culture, art.
he argued that, for an anthropologist, art must be treated as person-like because it represents both sources of and targets for agency.
now, what applies to art also relates to our hand-crafted bodies, to the bespoke objects and artefacts that we make, as well as to things that we find in the world.
consequently artefacts and things draw their symbolic force from association with agents and in particular with the relationships they have with our bodies [\dots]
it is also misleading to present these relationships as lopsided.
viewed rationally, the respect:hate relationship i have with my computer should be a one-way process in a simple network of domination between subject and object.
however as knappett puts it

\vspace{0.2cm}

``neither is it the case that people have the upper hand in these networks, merely manipulating materials as they see fit; agency is distributed between humans and nonhumans such that we have to tackle them symmetrically rather than assume from the outset an unbalanced relationship.''

\vspace{0.5cm}

(clive gamble. origins and revolutions: human identity in earliest prehistory. cambridge university press. 2007)


\end{center}

\end{document}


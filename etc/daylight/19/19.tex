\documentclass[12pt,a4paper,british,landscape]{article}

\usepackage{anyfontsize}


\renewcommand*\ttdefault{cmvtt}
\renewcommand{\familydefault}{\ttdefault}

\usepackage[OT2,T1]{fontenc}

\renewcommand\labelitemi{---}


\usepackage{graphicx}
\usepackage[
    a4paper,
    bindingoffset=0cm,
    left=1cm,
    right=1cm,
    top=1.1cm,
    bottom=0.2cm,
    footskip=0cm
]{geometry}

\frenchspacing              % Better looking spacings after periods
\pagestyle{empty}           % No pagenumbers/headers/footers



\begin{document}

{\fontsize{0.915cm}{1cm}\selectfont
``in the early years of the sixth century, st. benedict formulated a code of law for organising monastic life which became the model for all subsequent monastic rules [\dots].
this daily routine was regulated by the seven periods of prayer [\dots] which were intended to act as reminders of the passion [\dots].
acoustic signals (\emph{signa}) were employed to announce the hours of the day, and the most common instrument for announcing the canonical hours was the bell.
the ringing of the bell was a serious matter within the fortress-like walls of the monastery [\dots].
this complete authority assigned to the bell in the monastic milieu established a strong precedent for the centralised control of time that would be exercised centuries later in factories and schools – and also, as we will see, on colonial missions.
as in factories and schools, the bell’s tolls were given precedence over all other tasks and activities in the monastery:
`as soon as the signal for the divine office is heard', the rule ordered, `the brethren must leave whatever they have been engaged in doing, and hasten with all speed, but with dignity.'
this system resulted in collective punctuality becoming the specific focus of monastic life.''

\vspace{0.25cm}

\begin{center}
(giordano nanni. the colonisation of time. manchester. 2012)
\end{center}

}


\end{document}

